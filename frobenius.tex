\documentclass[12pt, a4paper]{article}
\usepackage[T1]{fontenc}
\usepackage[utf8]{inputenc}
\usepackage[english]{babel}
\usepackage{comment}
\usepackage{amsmath,amsthm, amssymb, bm, amsfonts}
\numberwithin{equation}{section}
\usepackage{tikz}
\usepackage{tikz-cd}
\usetikzlibrary{matrix, positioning}
\usepackage{enumerate}
\theoremstyle{definition}
\newtheorem{definition}{Definition}[section]
\newtheorem{example}[definition]{Example}
\theoremstyle{plain}
\newtheorem{lemma}[definition]{Lemma}
\newtheorem{proposition}[definition]{Proposition}
\newtheorem{corollary}[definition]{Corollary}
\newtheorem{Conjecture}[definition]{Conjecture}
\newtheorem{theorem}[definition]{Theorem}
\theoremstyle{remark}
\newtheorem{remark}[definition]{Remark}
\title{Reflexive modules}
\author{Javier Herrero, Javier San Martin Martinez, Maša Žaucer}
\date{\today}

\begin{document}

\maketitle


\begin{abstract}
\end{abstract}

\section{Introduction}
Through this document we study necessary and sufficient conditions for a module to be reflexive over a Nakayama algebra. We recall that a module is reflexive over $R$ if $M^{**}\cong M$ where $M^{**}:=Hom_R(Hom_R(M,R),R).$ We recall that is sufficient to check that the module $M$ is torsionless and the dimension coincide.
\begin{remark}
    For classifying reflexive modules is enough to study indecomposable modules since any other module decomposes into direct sum of indecomposables and taking double duals commutes with direct sums, i.e. $\oplus_i M_i^{**}\cong (\oplus_iM_i)^{**}.$
\end{remark}
For a linear Nakayama algebra $A$ we know that the indecomposable modules are given by $s_{i,k}:=e_iA/J^kA$ where $J$ is the Jacobson Radical. This indecomposables fit nicely in the Auslander-Reiten quiver which we recall is determined by Dyck paths, using the Kupisch series which determines their Auslander-Reiten quiver. We want to use this combinatorial representation to give an explicit method to compute duals and eventually classify the indecomposable modules using the (co)Kupisch series. 

\section{Combinatorial method to compute duals}
We ilustrate our method with the following Auslander-Reiten quiver corresponding to the Nakayama algebra with Kupisch series $[4,3,2,3,2,1].$

Moreover we can determine precisely the dual using the following remark and proposition.
\begin{remark}
    $Hom(s_{i,k},A)\cong Ae_i/J^wA$ where $w$ is the dimension of the dual. This means that the dual is indecomposable. We can further obtain more information about the dual by regarding the projective module $s_{i',k'}$ for the smallest $i'$ such that $Hom(s_{i,k},s_{i',k'})\neq 0$ then is clear that $Hom(s_{i,k},R)\cong s_{i',w}.$ 
\end{remark}
With this it is clear that in order to compute the dual we should regard the number of projectives in the Dyck path, to which there exist non trivial maps from our module. We should find the one with the lowest $i$ which is the one more on the right in the picture. We realize this by the following geometrical construction. 

If we want to compute the dual of $s_{i,k}$ we draw a rectangle with one of the vertices in $s_{i,k}.$ We draw a line in direction right down until we touch the $x$ axis and from there up right until the quiver ends. We finish the rectangle by drawing two parallel lines in the obvious way. By the properties of the Auslander-Reiten quiver the amount of projectives in the square corresponds to the dimension of the dual. We also know by the explanation above that the dual should have same first index as the vertex in the right top position.
\begin{proposition}
    For a Nakayama algebra with Kupisch series $c_i=\text{dim}(e_iA)$ and $d_i=\text{dim}(Ae_i)$, the dual of $s_{i,k}$ is $s_{i-d_i+1,k+d_i-d_{i+k}}.$
\end{proposition}
\begin{proof}
    We first compute the vertices of the square. The right lower is $s_{i,1}$ we then consider the top right vertex which would be $s_{i-d_{i}+1,d_{i}}.$ From this it follows that the first index is indeed $i-d_i+1.$ We compute the dimension. Notice that in each diagonal lies $d_{i}-d_{i+1}+1$ elements. We then consider the $k$ diagonals and get:
    $$\sum_{j=0}^{k-1} d_{i+j}-d_{i+j+1}+1=k+d_i-d_{i+k}$$
\end{proof}
We remark that indeed this same method can be iterated just realizing that the Auslander-Reiten quiver associated to The Nakayama algebra by right multiplication is the same but with the arrows reversed. 

We study now the specialization of the formula above for the case of simple mpodules.


\section{Computation of the dimension of the double dual}
Take a module $M_{i,k}$, where $i$ presents the first non-zero index in our chosen module and $k$ its dimension. 
In order to compute the dual, we need to count all the projective modules, reachable from $M_{i,k}$. 
Since we can only move towards the right, by line $j$ we refer to the diagonal starting at index $j$ and going to the right. The length of line $j$ is $d_j$.
The reachable projectives can lie on lines $j = i, i+1, \dots, i+k-1$.
On line $j$ there are exactly $d_j - d_{j+1} + 1$ projective modules, where $d_j$ is the number of modules on line $j$.
Thus the number of projective modules reachable from $M_{i,k}$ is given by
\begin{equation}
\label{eq:reachable}
\sum_{j=i}^{i+k-1} (d_j - d_{j+1} + 1) = d_i - d_{i+k} + k,
\end{equation}
and is equal to the dimension of the dual $M^*_{i,k}$.
Since the dual lies on the line extending the upper bound of the rectangle of reachable modules, the first index is given by $i - d_i + 1$.
(From index $i$ we go up $d_i - 1$ times to get the right most vertex of the triangle. To reach the dimension $1$ again we then need to go down $d_i - 1$ times, where each step down increases the index by $1$.)

The first dual is thus the module $M_{i - d_i + 1, d_i - d_{i+k} + k}$.

To compute the second dual, we reverse the arrows and move to the left on the same Dyck path. By line we now mean the diagonal starting at index $j$ and going to the left and the length of line $j$ is $c_j$.
For convinience we will first compute the dual of an arbitrary module $M_{l,m}$ again, this time in the other direction. We will then enter the indices of the first dual obtained above.

To get the dimension of the dual, by duality between projectives and injectives, we need to count the injective modules reachable from $M_{i,k}$.
They lie on lines $j = l, l+1, \dots, l+m-1$ again and on line $j$ there are $c_j - c_{j+1} + 1$ of them.
We get that the dimension of the dual is given by
\begin{equation}
\label{eq:injectives}      
\sum_{j=l}^{l+m-1} (c_j - c_{j+1} + 1) = c_{l+m-1} - c_{l-1} + m.
\end{equation}

It remains to compute the first index of the dual.
First observe that from the module $M_{l,m}$ we can reach the dimension $1$ by going down $(m - 1)$-times. The left-most vertex then lies on the same line as the lower vertex and thus has index $l + m - 1$.
From the left most vertex we need to make additional steps down, until we reach the desired dimension of the dual.
The number of steps is $(c_{l+m-1} - dim)$, where $dim$ denotes the dimension of the dual.
Each step down increases the index by $1$, therefore we need to increase the index of left most vertex by $c_{l+m-1} - c_{l+m-1} + c_{l-1} - m = c_{l-1} - m$, where we took into account the dimension computed above.

We get that the first index of the dual is given by 
$$l + m - 1 + c_{l-1} - m = l + c_{l-1} - 1.$$

To get the indices of double dual, we just need to set 
\begin{align*}
    l &= i - d_i + 1, \\
    m &= d_i - d_{i+k} + k.
\end{align*}

We get that the double dual of module $M_{i,k}$ is given by the indices
\begin{align*}
    &i - d_i + c_{i - d_i}, \\
    &k + d_i - d_{i+k} + c_{i+k-{d_{i+k}}} - c_{i-d_i}.
\end{align*}

In order for the module $M_{i,k}$ to be reflexive, we need to have $i - d_i + c_{i - d_i} = i$ and $k + d_i - d_{i+k} + c_{i+k-{d_{i+k}}} - c_{i-d_i} = k$.
The first equation gives us $c_{i - d_i} = d_i$ and the second one, taking into account also the first equality, gives us $d_{i+k} = c_{i+k-{d_{i+k}}}$.



\bibliographystyle{plain} 
\bibliography{refs}

\end{document}