\documentclass[12pt, a4paper]{article}
\usepackage[T1]{fontenc}
\usepackage[utf8]{inputenc}
\usepackage[english]{babel}
\usepackage{comment}
\usepackage{amsmath,amsthm, amssymb, bm, amsfonts}

\title{Reflexive modules}
\author{Javier Herrero, Javier San Martin Martinez, Maša Žaucer}
\date{\today}

\begin{document}

\maketitle


\begin{abstract}
\end{abstract}

\section{Introduction}

\section{Computation of the dimension of the double dual}
Take a module $M_{i,k}$, where $i$ presents the first non-zero index in our chosen module and $k$ its dimension. 
In order to compute the dual, we need to count all the projective modules, reachable from $M_{i,k}$. 
Since we can only move towards the right, by line $j$ we refer to the diagonal starting at index $j$ and going to the right. The length of line $j$ is $d_j$.
The reachable projectives can lie on lines $j = i, i+1, \dots, i+k-1$.
On line $j$ there are exactly $d_j - d_{j+1} + 1$ projective modules, where $d_j$ is the number of modules on line $j$.
Thus the number of projective modules reachable from $M_{i,k}$ is given by
\begin{equation}
\label{eq:reachable}
\sum_{j=i}^{i+k-1} (d_j - d_{j+1} + 1) = d_i - d_{i+k} + k,
\end{equation}
and is equal to the dimension of the dual $M^*_{i,k}$.
Since the dual lies on the line extending the upper bound of the rectangle of reachable modules, the first index is given by $i - d_i + 1$.
(From index $i$ we go up $d_i - 1$ times to get the right most vertex of the triangle. To reach the dimension $1$ again we then need to go down $d_i - 1$ times, where each step down increases the index by $1$.)

The first dual is thus the module $M_{i - d_i + 1, d_i - d_{i+k} + k}$.

To compute the second dual, we reverse the arrows and move to the left on the same Dyck path. By line we now mean the diagonal starting at index $j$ and going to the left and the length of line $j$ is $c_j$.
For convinience we will first compute the dual of an arbitrary module $M_{l,m}$ again, this time in the other direction. We will then enter the indices of the first dual obtained above.

To get the dimension of the dual, by duality between projectives and injectives, we need to count the injective modules reachable from $M_{i,k}$.
They lie on lines $j = l, l+1, \dots, l+m-1$ again and on line $j$ there are $c_j - c_{j+1} + 1$ of them.
We get that the dimension of the dual is given by
\begin{equation}
\label{eq:injectives}      
\sum_{j=l}^{l+m-1} (c_j - c_{j+1} + 1) = c_{l+m-1} - c_{l-1} + m.
\end{equation}

It remains to compute the first index of the dual.
First observe that from the module $M_{l,m}$ we can reach the dimension $1$ by going down $(m - 1)$-times. The left-most vertex then lies on the same line as the lower vertex and thus has index $l + m - 1$.
From the left most vertex we need to make additional steps down, until we reach the desired dimension of the dual.
The number of steps is $(c_{l+m-1} - dim)$, where $dim$ denotes the dimension of the dual.
Each step down increases the index by $1$, therefore we need to increase the index of left most vertex by $c_{l+m-1} - c_{l+m-1} + c_{l-1} - m = c_{l-1} - m$, where we took into account the dimension computed above.

We get that the first index of the dual is given by 
$$l + m - 1 + c_{l-1} - m = l + c_{l-1} - 1.$$

To get the indices of double dual, we just need to set 
\begin{align*}
    l &= i - d_i + 1, \\
    m &= d_i - d_{i+k} + k.
\end{align*}

We get that the double dual of module $M_{i,k}$ is given by the indices
\begin{align*}
    i - d_i + c_{i - d_i}, \\
    k + d_i - d_{i+k} + c_{i+k-{d_{i+k}}} - c_{i-d_i}.
\end{align*}

In order for the module $M_{i,k}$ to be reflexive, we need to have $i - d_i + c_{i - d_i} = i$ and $k + d_i - d_{i+k} + c_{i+k-{d_{i+k}}} - c_{i-d_i} = k$.
The first equation gives us $c_{i - d_i} = d_i$ and the second one, taking into account also the first equality, gives us $d_{i+k} = c_{i+k-{d_{i+k}}}$.



\bibliographystyle{plain} 
\bibliography{refs}

\end{document}